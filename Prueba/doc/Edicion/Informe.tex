\documentclass{article}
\usepackage{latexsym}
\usepackage[utf8]{inputenx}
\usepackage[spanish]{babel}
\usepackage{graphicx}
\usepackage{anysize}
\usepackage{amsmath}
\usepackage{amssymb}
\usepackage{float}
\usepackage{fancyhdr}
\setlength{\skip\footins}{5cm}
\usepackage{lscape}
\usepackage{verbatim}
\usepackage{moreverb}
\usepackage{url}
\usepackage{enumitem}
\usepackage{multicol}
\usepackage{pifont}
\let\verbatiminput=\verbatimtabinput
\usepackage[nottoc,numbib]{tocbibind}
\setcounter{tocdepth}{4}
\setcounter{secnumdepth}{4}

\marginsize{2cm}{2cm}{.5cm}{3cm} 
\pagestyle{myheadings}
\renewcommand{\headrulewidth}{0.5pt}
\begin{document}

\begin{titlepage}

\newcommand{\HRule}{\rule{\linewidth}{0.5mm}} % Defines a new command for the horizontal lines, change thickness here

\center % Center everything on the page
 
%----------------------------------------------------------------------------------------
%	HEADING SECTIONS
%----------------------------------------------------------------------------------------

\textsc{\LARGE Universidad De Buenos Aires}\\[1.5cm] % Name of your university/college
\textsc{\Large Facultad De Ingeniería}\\[0.5cm] % Major heading such as course name
\textsc{\large 75.52 Taller de Programaci\'on II}\\[0.5cm] % Minor heading such as course title

%----------------------------------------------------------------------------------------
%	TITLE SECTION
%----------------------------------------------------------------------------------------

\HRule \\[0.4cm]
{ \huge \bfseries Recursos}\\ Capa de Negocios\\[0.4cm] % Title of your document
\HRule \\[1.5cm]
 
%----------------------------------------------------------------------------------------
%	AUTHOR SECTION
%----------------------------------------------------------------------------------------

% If you don't want a supervisor, uncomment the two lines below and remove the section above
\Large \emph{Integrantes:}\\

Hugo \textsc{Chavar} - 90541\\ % Your name
Dami\'an \textsc{Manoff} - 93169\\ % Your name
Yamila \textsc{Glinsek} - 93219\\ % Your name
Andr\'es \textsc{Sanabria} - 93403\\[5cm] % Your name

\textit{capanegocio.recursos@yahoo.com.ar}
%----------------------------------------------------------------------------------------
%	DATE SECTION
%----------------------------------------------------------------------------------------

{\large \text \em \today }\\[3cm] % Date, change the \today to a set date if you want to be precise
%{10 de Septiembre de 2013}
 
%----------------------------------------------------------------------------------------

\vfill % Fill the rest of the page with whitespace

\end{titlepage}
\tableofcontents
\newpage
\section{Introducci\'on}
Recursos da referencia a lo que podemos traducir como Archivos, Links y Encuestas.\\
Los servicios que brindamos son:
\begin{enumerate}
	\item Posibilitar la subida y descarga de archivos al sistema.
	\item Crear encuestas, que pueden ser o no \emph{Evaluadas}.
	\item Permitir el acceso a responder encuestas y realizar su evaluaci\'on seg\'un el criterio establecido por el creador de la misma.
	\item Linkear p\'aginas externas y/o internas de la web.
\end{enumerate}
\markboth{Capa Negocio}{Capa Negocio- Recursos/Materiales - Grupo 2 - Tp Taller II - v 1.1}	
\section{Definiciones}
\begin{description}
	\item Cosas que ya se han definido hasta el momento por la interacci\'on con los grupos y docentes:
	\renewcommand{\labelitemi}{\ding{85}} 
	\begin{itemize}
		\item Se utilizar\'a como IDE \emph{Eclipse} y plataforma \emph{Java 7}.
		\item La comunicaci\'on entre capas se realiza por Web Services.
		\item Los archivos van a ser pasados entre capas como multi-part (o similar) a trav\'es de dichos Web Services.
		\item Tanto con la capa de arriba (Presentaci\'on) como la de abajo (Integraci\'on) implementar\'an la interfaz en XML.
		\item Los archivos ser\'an pasados de nuestra capa a la  \emph{capa de Integraci\'on}  para que \'esta trate con el FileSystem o con la \emph{capa de Acceso a datos} como ha de ser almacenado. Queda por definirse si es necesario una conversaci\'on previa entre capas antes de la transferencia del archivo.
%\item 

	\end{itemize}

\subsection{Archivos}
	Para el manejo de archivos se tendr\'a la siguiente clase:\\
	
	\includegraphics[scale=0.7]{Archivo.png}	
	
	La cual tiene como atributo:\\
	\begin{itemize}
		\item Path: que es una cadena que cuenta con la direccion en la cual se encuentra guardado en el file system el archivo. En principio se define una longitud maxima de 100 caracteres para este campo.\textit{Este campo es fundamental para la capa de Persistencia}.\\
		\item Tamaño: es el peso del archivo en Bytes, por lo tanto es solo num\'erico. Consideraremos un maximo de 100 MB para cada archivo. Por lo tanto la longitud maxima del campo ser\'a 10 caracteres.\\
		\item Tipo se refiere a la extensi\'on del Archivo. Considerando un video mpeg4, la longitud de la cadena ser\'a de 5 caracteres.\\
		\item Nombre es la identificaci\'on del archivo que ser\'a una cadena de 40 caracteres como m\'aximo.\\
	\end{itemize}

\subsection{Encuentas}
La idea de encuesta es algo del siguiente tipo:

\includegraphics[scale=0.4]{EncuestaFoto.png}	

En donde claramente se pueden distinguir 3 tipos diferentes de preguntas:
\begin{enumerate}
\item Preguntas que solo admiten una opci\'on como respuesta.
\item Preguntas que admiten multiples opciones como respuesta.
\item Preguntas cuya respuesta es a completar.
\end{enumerate}

Adem\'as existir\'an dos tipos de encuestas:
\begin{enumerate}
\item Encuestas Evaluadas
\item Encuestas No Evaluadas.
\end{enumerate}

Con respecto al primer tipo de encuestas solo podremos admitir preguntas del tipo 1 y 2 ya que se dificulta la tarea de evaluar respuestas a completar.\\
Las encuestas del segundo tipo admitir\'an todo tipo de preguntas.
\end{description}

\markboth{Capa Negocio}{Capa Negocio- Recursos/Materiales - Grupo 2 - Tp Taller II - v 1.1}	
\section{Diagrama de Clases}
	\includegraphics[scale=0.4]{Diagramadeclase4.png}
\markboth{Capa Negocio}{Capa Negocio- Recursos/Materiales - Grupo 2 - Tp Taller II - v 1.1}	
\pagebreak
	

\section{Interfaces}
	\begin{description}
		\item[Provistas a la capa de Presentaci\'on ] \
		\renewcommand{\labelitemi}{\ding{105}} 
		\begin{itemize}
			\item \emph{OBTENER LISTA RECURSOS}
			\begin{description}
				\item[INPUT] ID-AMBIENTE
				\item[OUTPUT] LISTA-DE [ID-RECURSO, DESCRIPCION, TIPO]\\
			\end{description}
			\item \emph{OBTENER RECURSO}
			\begin{description}
				\item[INPUT] ID-AMBIENTE, ID-RECURSO, ID-USUARIO
				\item[OUTPUT] LINK-AL-RECURSO \'o ARCHIVO \'o ACCESO-A-ENCUESTA\\
			\end{description}
			\item \emph{AGREGAR ARCHIVO}
			\begin{description}
				\item[INPUT] ID-AMBIENTE, DESCRIPCI\'ON, ARCHIVO
				\item[OUTPUT] MENSAJE DE CONFIRMACI\'ON\\
			\end{description}
			\item \emph{AGREGAR LINK}
			\begin{description}
				\item[INPUT] ID-AMBIENTE, DESCRIPCI\'ON, LINK
				\item[OUTPUT] MENSAJE DE CONFIRMACI\'ON (\emph{Coordinar con presentaci\'on que se requiere})\\
			\end{description}
			\item \emph{AGREGAR ENCUESTA}
			\begin{description}
				\item[INPUT] ID-AMBIENTE, DESCRIPCI\'ON, ENCUESTA (\emph{Coordinar con presentaci\'on forma de pasarlo})
				\item[OUTPUT] MENSAJE DE CONFIRMACI\'ON\\
			\end{description}
			\item \emph{GUARDAR ENCUESTA RESPONDIDA}
			\begin{description}
				\item[INPUT] ID-ENCUESTA, ID-USUARIO, LISTA-RESPUESTAS
				\item[OUTPUT] MENSAJE DE CONFIRMACI\'ON (\emph{Coordinar con presentaci\'on})\\
			\end{description}
			\item \emph{OBTENER ENCUESTA RESPONDIDA}
			\begin{description}
				\item[INPUT] ID-ENCUESTA, ID-USUARIO
				\item[OUTPUT] LISTA-RESPUESTAS\\
			\end{description}
			\item \emph{BORRAR RECURSO}
			\begin{description}
				\item[INPUT] ID-AMBIENTE, ID-RECURSO
				\item[OUTPUT] MENSAJE DE CONFIRMACI\'ON(\emph{Coordinar con presentaci\'on si requieren algun dato})\\
			\end{description}
		\end{itemize}
%		\renewcommand{\labelitemi}{\ding{118}} 
%		\item[Requeridas a la capa de Integraci\'on \emph{( incompleto)}] \
%		\\Nota:
%		\emph{ En un principio, no necesitaremos que la capa de integraci\'on haga joins de tablas de Base de Datos, ya que todo va a ser requerido por ID.}
%		\begin{itemize}
%			\item \emph{OBTENER LISTA RECURSOS}
%			\begin{description}
%				\item[INPUT] ID-AMBIENTE
%				\item[OUTPUT] LISTA-DE [ID-RECURSO, DESCRIPCION, TIPO]
%			\end{description}
%			\item \emph{OBTENER RECURSO}
%			\begin{description}
%				\item[INPUT] ID-AMBIENTE, ID-RECURSO
%				\item[OUTPUT] LINK-AL-RECURSO
%			\end{description}
%			\item \emph{OBTENER ARCHIVO}
%			\begin{description}
%				\item[INPUT] LINK-AL-ARCHIVO
%				\item[OUTPUT] ARCHIVO
%			\end{description}
%			\item \emph{AGREGAR RECURSO}
%			\begin{description}
%				\item[INPUT] ID-AMBIENTE, DESCRIPCI\'ON, TIPO, DATOS (\emph{Difiere seg\'un tipo de recurso})
%				\item[OUTPUT] MENSAJE DE CONFIRMACI\'ON \'o LINK-AL-RECURSO
%			\end{description}
%			\item \emph{BORRAR RECURSO}
%			\begin{description}
%				\item[INPUT] ID-AMBIENTE, ID-RECURSO
%				\item[OUTPUT] MENSAJE DE CONFIRMACI\'ON
%			\end{description}
%		\end{itemize}
	\end{description}
	\markboth{Capa Negocio}{Capa Negocio- Recursos/Materiales - Grupo 2 - Tp Taller II - v 1.1}
\end{document}
